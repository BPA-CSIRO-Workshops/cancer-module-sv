% Define the top matter
\setModuleTitle{Structural Variant Analysis}
\setModuleAuthors{%
  Erdahl Teber \mailto{eteber@cmri.org.au}
}
\setModuleContributions{%
  Mathieu Bourgey \mailto{mathieu.bourgey@mcgill.ca}%
  Sonika Tyagi\mailto{sonika.tyagi@agrf.org.au}
}

%  Start: Module Title Page
\chapter{\moduleTitle}
\newpage
% End: Module Title Page

\section{Key Learning Outcomes}
By the end of the structural variant (SV) detection practical course participants will:
\begin{itemize}
  \item Have discussed key fundamentals on how paired-end mappings and split-read/soft-clipped read patterns are used in detecting deletions, tandem duplicates, inversions and translocations 
  \item Know what important quality control checks need to be evaluated prior to structural variant calling. 
  \item Have run DELLY on a subset of whole genome next generation sequencing data pertaining to a single human tumour with a matched normal control.
  \item Be able to filter high confidence SV predictions. 
  \item Have gained basic knowledge to interpret the VCF output provided by DELLY.
  \item Have used their understanding of distinct SV paired-end mapping and soft-clipped read patterns to visually verify DELLY predicted SVs using IGV.
\end{itemize}

\section{Resources You'll be Using}
 
\subsection{Tools Used}
\begin{description}[style=multiline,labelindent=0cm,align=left,leftmargin=0.5cm]
  \item[]\hfill\\
  	\url{}
  \item[]\hfill\\
  	\url{}
\end{description}

\section{Useful Links}
 
\begin{description}[style=multiline,labelindent=0cm,align=left,leftmargin=0.5cm]
  \item[SAM Specification]\hfill\\
    \url{http://samtools.sourceforge.net/SAM1.pdf}
  \item[Explain SAM Flags]\hfill\\
    \url{http://picard.sourceforge.net/explain-flags.html}
\end{description}

\subsection{Sources of Data}
% TODO more specific about how the data was derived for this module
  \url{http://sra.dnanexus.com/studies/ERP001071}

\clearpage

\section{Alignment Quality Control}

\begin{information}
For structural variant calling several alignment quality control metrics should be evaluated. All paired-end mapping methods heavily rely on the insert size distribution. GC-content biases are important as it can impact read-depths. DELLY generates read-depth ratios between tumour and control samples. The percentage of mapped reads, singletons, duplicates and properly paired reads are additional metrics you should evaluate prior to any structural variant calling. These statistics vary largely by protocol and hence, it is usually best to compare multiple different sequencing runs using the same against each other to highlight outliers. 

It is recommended that Picard module commands \texttt{CollectInsertSizeMetrics} and \texttt{CollectGcBiasMetrics}, and \texttt{samtools flagstat} command be used. 

\end{information}

\section{Prepare the Environment}

\begin{note}
As a quick introduction we will do a structural variant analysis using a single immortal cancer cell line and its control genome (mortal parental cells). Total execution time to run the DELLY structural discovery calling program for a matched normal tumour pair will vary depending on the read coverage and the size of the genome. As a guide, it can take approximately 10 to 50 hours (translocation predictions taking the longest), for a matched normal tumour pair (each 40-50x coverage) running on 2 cpus on a server with sufficient RAM.

The bam files we will be working on are a subset of the original WGS bam files, limited to specific chromosomal regions to speed up the analysis and to meet the time constraints for this practical. 
\end{note}

\begin{information}
Firstly, we will use shell variables to help improve the readability of commands and streamline scripting. Each distinct variable will store a directory path to either, the input WGS bam files, hg19 reference, programs or output.

\end{information}

\begin{steps}
Open the Terminal.

First, go to the right folder, where the data are stored.
\begin{lstlisting}
cd ~/sv
ls
mkdir <YourFirstName>
cd <YourFirstName>

export DS=/home/trainee/sv/data
export OP=/home/trainee/sv/delly_output
export RF=/home/trainee/sv/reference_data
export SF=/home/trainee/sv/variantFiltering/somaticVariants
export BR=/home/trainee/snv/Applications/igv
export CZ=/home/trainee/sv/converter

\end{lstlisting}

\end{steps}

\section{Somatic Structural Variant Discovery using DELLY}

\begin{information}
In order to generate ‘putative’ somatic SVs it is crucial to account for germline SVs. To facilitate, DELLY requires the joint input of a match normal control and the cancer aligned sequencing data (bam files).
\end{information}

\begin{steps}

\begin{lstlisting}
delly -t DEL -x \$RF/hg19.excl -o del.vcf -g \$RF/hg19.fa \$DS/cancer_cell_line.bam \$DS/control.bam
delly -t DUP -x \$RF/hg19.excl -o dup.vcf -g \$RF/hg19.fa \$DS/cancer_cell_line.bam \$DS/control.bam
delly -t INV -x \$RF/hg19.excl -o inv.vcf -g \$RF/hg19.fa \$DS/cancer_cell_line.bam \$DS/control.bam
delly -t TRA -x \$RF/hg19.excl -o tra.vcf -g \$RF/hg19.fa \$DS/cancer_cell_line.bam \$DS/control.bam

\end{lstlisting}

We have used the following arguments for the indexing of the genome.
\begin{description}[style=multiline,labelindent=0cm,align=right,leftmargin=\descriptionlabelspace,rightmargin=1.5cm,font=\ttfamily]
  \item[DEL] = 
  \item[DUP] = 
  \item[INV] = 
  \item[TRA] = 
  \item[-o] = 
  \item[-g] = 
\end{description}

\begin{warning}
  \begin{lstlisting}
 %Any pre-computed command can go here
  \end{lstlisting}
\end{warning}

This command will output x files that constitute the 

\section{Somatic Structural Variant Discovery using DELLY}

\begin{information}
A VCF file has multiple header lines starting with the hash # sign. Below the header lines is one record for each structural variant. The record format is described in the below table:

1	CHROM	Chromosome name
2 	POS 		1-based position. For an indel, this is the position preceding the indel.
3 	ID 		Variant identifier. 
4	REF		Reference sequence at POS involved in the variant. 
5	ALT		Comma delimited list of alternative sequence(s).
6	QUAL		Phred-scaled probability of all samples being homozygous reference.
7	FILTER	Semicolon delimited list of filters that the variant fails to pass.
8	INFO		Semicolon delimited list of variant information.
9	FORMAT	Colon delimited list of the format of sample genotypes in subsequent fields.
10+ 	Samples	Individual genotype information defined by FORMAT.

\begin{table}[H]
  \centering
  \caption{VCF output description}
    \begin{tabular}{ll}
    \toprule
    Column & Field & Description \\
    \midrule
    1 & CHROM & Chromosome name\\
    2 & POS & 1-based position. For an indel, this is the position preceding the indel \\
    3 & ID & Variant identifier\\
    4 & REF & Reference sequence at POS involved in the variant\\
    5 & ALT & Comma delimited list of alternative sequence(s)\\
    6 & QUAL & Phred-scaled probability of all samples being homozygous reference\\
    7 & FILTER & Semicolon delimited list of filters that the variant fails to pass\\
    8 & INFO & Semicolon delimited list of variant information\\
    9 & FORMAT & Colon delimited list of the format of sample genotypes in subsequent fields\\
    10+ & &Individual genotype information defined by FORMAT \\
    \bottomrule
    \end{tabular}
  \label{tab:vcfoutput}
\end{table}


\end{information}
\begin{steps}
You can look at the header of the vcf file using grep, ‘-A 1’ includes the first structural variant record in the file:

\begin{lstlisting}
grep "^#" -A 1 del.vcf
\end{lstlisting}
\end{steps}

\begin{note}

The INFO field holds structural variant site information whereas all genotype information (annotated as per the FORMAT fields) is provided in the sample column. Reference supporting reads are compared to alternative supporting reads and mapping qualities are used to compute genotype likelihoods (GL) for homozygous reference (0/0), heterozygous reference (0/1) and homozygous alternate (1/1) (GT). The final genotype (GT) is simply derived from the best GL and GQ is a phred-scaled genotype quality reflecting the confidence in this genotype. If GQ<15 the genotype is flagged as LowQual. The genotyping takes into account all paired-ends with a mapping quality greater 20 by default. 

The INFO field provides information on the quality of the SV prediction and breakpoints. If you browse through the vcf file you will notice that a subset of the DELLY structural variant predictions have been refined using split-reads. These precise variants are flagged in the vcf info field with the tag ‘PRECISE’. To count the number of precise and imprecise variants you can simply use grep.

\end{note}

\begin{steps}
\begin{lstlisting}
grep -c -w "PRECISE" *.vcf
grep -c -w "IMPRECISE" *.vcf
\end{lstlisting}
\end{steps}


\begin{information}
DELLY clusters abnormal paired-ends and every single cluster gives rise to an IMPRECISE SV call. For every IMPRECISE SV call an attempt is made to identify supporting split-reads/soft-clipped reads DELLY then computes a consensus sequence (INFO:CONSENSUS) out of all split-read candidates and then aligns this consensus sequence to the reference requiring at least -m many aligned bases to the left and right (default is 13). INFO:PE is the number of supporting paired-ends. INFO:CT refers connection types (CT), which indicates the order and orientation of paired-end cluster mappings (e.g. 3to3 for 3' to 3' and 5to5 for 5' to 5'). Values can be 3to5, 5to3, 3to3 or 5to5. Different names exist for these connection types in the literature, head-to-head inversions, tail-to-tail inversions, and so on. The consensus alignment quality (SRQ) is a score between 0 and 1, where 1 indicates 100% identity to the reference. Nearby SNPs, InDels and micro-insertions at the breakpoint can lower this score but only for mis-assemblies it should be very poor. DELLY currently drops consensus alignments with a score < 0.8 and then falls back to an IMPRECISE prediction. 

SVs are flagged as FILTER:LowQual if PE<3 OR MAPQ<20 (for translocations: PE<5 OR MAPQ<20), otherwise, the SV results in a FILTER:PASS. PRECISE variants will have split-read support (SR>0).

\end{information}

\section{Somatic Structural Variant Filtering}

\begin{information}
Please note that this vcf file contains germline and somatic structural variants but also false positives caused by repeat induced mis-mappings or incomplete reference sequences. As a final step we have to use the structural variant site information and the cancer and normal genotype information to filter a set of confident somatic structural variants. DELLY ships with a somatic filtering python script. For a set of confident somatic calls one could exclude all structural variants <400bp, require a minimum variant allele frequency of 10%, no support in the matched normal and an overall confident structural variant site prediction with the VCF filter field being equal to PASS.

\end{information}
\begin{steps}
\begin{lstlisting}
python \$SF/somaticFilter.py -t DEL  -T cancer_cell_line -N control -v del.vcf -o del.filt.vcf -a 0.1 -m 400 -f
python \$SF/somaticFilter.py -t DUP -T cancer_cell_line -N control -v dup.vcf -o dup.filt.vcf -a 0.1 -m 400 -f
python \$SF/somaticFilter.py -t INV  -T cancer_cell_line -N control -v inv.vcf -o inv.filt.vcf -a 0.1 -m 400 -f
python \$SF/somaticFilter.py -t TRA -T cancer_cell_line -N control -v tra.vcf -o tra.filt.vcf -a 0.1 -m 400 -f
\end{lstlisting}
\end{steps}

We have used the following arguments for .
\begin{description}[style=multiline,labelindent=0cm,align=right,leftmargin=\descriptionlabelspace,rightmargin=1.5cm,font=\ttfamily]
  \item[-t] = 
  \item[-T] = 
  \item[-N] = 
  \item[-v] = 
  \item[-o] = 
  \item[-a] = 
  \item[-m] = the value 400, means ignore SVs that are <400 bp in size and -a 0.1, refers to a minimum variant frequency of 10%. 
\end{description}


\begin{steps}
Using VCFtools we can merge all somatic structural variants together in a single vcf file.
\begin{lstlisting}
vcf-concat del.filt.vcf dup.filt.vcf inv.filt.vcf tra.filt.vcf | vcf-sort > somatic.sv.vcf
\end{lstlisting}
\end{steps}

\begin{information}
For large VCF files you should also zip and index them using bgzip and tabix. Please run the below commands to meet the requirements for visualising somatic structural variants using IGV.  
\end{information}

\begin{steps}
\begin{lstlisting}
bgzip somatic.sv.vcf
tabix somatic.sv.vcf.gz
\end{lstlisting}
\end{steps}



\section{Visualisation of Somatic Structural Variants}

\begin{information}
The final step will be to browse some of these somatic structural variants in IGV and to visually verify the reliability of the calls. VCF file structure was designed by the 1000Genomes Project consortium with considerable focus being placed on single-nucleotide variants and InDels. It is arguable whether the file format is easy to interpret for basic SVs and much less for complex SV formatting. To make it easier to see the breakpoints we will create a bed file. 

\end{information}
\begin{steps}
\begin{lstlisting}
\$CZ/dellyVcf2Bed.sh somatic.sv.vcf.gz > somatic.sv.bed
head somatic.sv.bed
\end{lstlisting}
\end{steps}

\begin{steps}
Load the IGV browser with hg19 genome specified with the \texttt{-g} option.
\begin{lstlisting}
\$BR/igv.sh -g \$RF/hg19.fa
\end{lstlisting}
\end{steps}

\begin{information}
Once IGV has started use ‘File’ and ‘Load from File’ to load the cancer_cell_line.bam. To dedicate more of the screen to read pile ups, right click on the left hand box that contains the ‘cancer_cell_line.bam Junctions’ then select ‘Remove Track’. Go to ‘View’ menu and select ‘Preferences’, then click on the Alignments tab and in the ‘Visibility range threshold (kb)’ text box, enter 600. This will allow you to increase your visibility of pile ups as you zoom out. Now look for check box for ‘Filter secondary alignments’ click on to ensure we do not see secondary alignments (alternate mapped position of a read) . Also ensure that ‘Show soft-clipped bases’ has been checked then click ‘OK’.
 
Then import ‘somatic.sv.bed’ from your working directory using ‘Regions’ and ‘Import Regions’. 

\end{information}


\begin{questions}
How many abnormal ‘paired-end read pairs’ (red coloured F/R oriented read pairs) can you see that spans the deletion region? Does this number coincide with the INFO:PE? 
\begin{answer}
Zoom into left breakpoint and tally the number of soft-clipped reads at both of this junctions.
19, YES
\end{answer}

How many abnormal split-reads (soft-clipped reads) did you observe? Clue INFO:SR.
\begin{answer}
Go to the RefSeq genes track at the bottom of IGV and right click to ‘Expanded’.
11
\end{answer}
Is the predicted deletion likely to have a deleterious impact on a gene? If so, what gene and exons are deleted?
\begin{answer}
ATRX, exons 2 to 25.
\end{answer}
Does this region appear to be completely removed from this cancer genome? How can you tell?
\begin{answer}
Yes, there is no read coverage within this deletion region relative to the control genome. 
\end{answer}
\end{questions}


% I thought these can go into the advanced excercise sections


\subsection{Verify translocation}
\begin{advanced}

\begin{lstlisting}
Select the duplication (chrX:45649874-45689322) from the ‘Region Navigator’. 

Highlight the abnormal paired-ends by clicking and selecting ‘Color alignments by’ and then switch to “insert size and pair orientation’. Also, invoke ‘Sort alignments by’ then select ‘start location’.

Zoom out until you can see all the purple reads at the junction.
\end{lstlisting}
\end{advanced}


\begin{questions}
What is the direction of the purple cluster of reads (indicates that mate reads are mapped to Chr15)? Is it pointing to the tail or head of Chr18? 
(HINT: Right click on to one of the purple coloured reads and select ‘View mate region in split-screen’. This will split the screen and display Chr15 on the left and place a red highlighted outline on both reads, to indicate the pairs. Right click and ‘Sort alignments by’ then select ‘start location’.)
\begin{answer}
Forward, towards the tail, or 3’ (+ive)
\end{answer}

What is the direction of the yellow cluster of reads (indicates that mate reads are mapped to Chr18)? Is it pointing to the tail or head of Chr15? 
If you wish to zoom in or out, first click inside of the chromosome ideogram panel, then ctrl- to zoom out and shift+ to zoom in. 
\begin{answer}
? Reverse, towards the head, or 5’ (+ive)
\end{answer}

How is the Chr15 and Chr18 fused (which one of the four translocation connection types)?  If you are uncertain then run a BLAT search using the INFO:CONSENSUS sequence (gedit  somatic.sv.vcf). https://genome.ucsc.edu/cgi-bin/hgBlat?command=start
\begin{answer}
RF, head to tail, or 5 to 3. Therefore, Chr18 left side is fused to Chr15 right side.
\end{answer}
Did DELLY predict a reciprocal translocation? How can you tell? 
\begin{answer}
No, as we would expect to observe a Chr18 right side fused to Chr15 left side, near the same breakpoints.  
\end{answer}
What gene structures is this translocation predicted impact? 
\begin{answer}
PARD6G on Chr15 and ADAMTSL3 on Chr18.
\end{answer}
What is one possible reason why there is no observable read coverage after Chr18 breakpoint? 
\begin{answer}
Chromosome loss.
\end{answer}
\end{questions}

\subsection{Verify tandem duplication}
\begin{advanced}
\begin{lstlisting}
Select the tandem duplication (chrX:45649874-45689322) from the ‘Region Navigator’.

Highlight the abnormal paired-ends by clicking and selecting ‘Color alignments by’ and then switch to “insert size and pair orientation’. Also, invoke ‘Sort alignments by’ then select ‘start location’.

Zoom out until you can see all the red paired-end reads spanning the two junctions. After that zoom in on the cluster of abnormal reads on the left junction and then right junction.
\end{lstlisting}
\end{advanced}

\begin{questions}
Which is the order and orientation of these paired-end reads (FR, RF, FF or RR)? 
\begin{answer}
RF
\end{answer}
What is the estimated read-depth ratio of the cancer_cell_line versus normal control (INFO:RDRATIO) over the duplicate region? 
\begin{answer}
3.3 (~3 x increased read depth) 
\end{answer}
\end{questions}


\subsection{Verify Inversion}
\begin{advanced}
\begin{lstlisting}

Type into the search box near at tool bar, Chr20:54834492 

Right click on main display and select ‘Group alignments by’ then switch on ‘paired-orientation’. Also, right click and ‘Sort alignments by’ then select ‘start location’. 

Zoom out until you can see all the red coloured cluster of reads near the breakpoint. 

Right click on to one of the red coloured reads and select ‘View mate region in split-screen’. This will split the screen and display the read mate on the right side. This will take you to the mate-reads near the second breakpoint. 
\end{lstlisting}
\end{advanced}


\begin{questions}
Which direction are the paired-end reads spanning (left or right spanning)?  
\begin{answer}
Right
\end{answer}
What is the estimated size of the inverted interval? 
\begin{answer}
55,408,660 – 54,834,492 = 574,168 bp
\end{answer}
\end{questions}

\section{Acknowledgements}

I would like to thank and acknowledge Tobias Rausch for his help, and allowing me to borrow and adapt his course material. Also, my thanks to Ann-Marie Patch for the helpful discussions in designing this practical. 


